\documentclass[a4,12pt]{article}
\usepackage[T1]{fontenc}
\usepackage{textcomp}
\usepackage[utf8]{inputenc}
\usepackage[english]{babel}
\usepackage{graphics}
\usepackage{graphicx}
\usepackage{epstopdf}
\usepackage{amsmath}
\usepackage{hyperref}
\usepackage{fancyvrb}
\usepackage{verbatim}
\usepackage{cmap}
%\renewcommand{\familydefault}{\sfdefault}
\textwidth=16cm
\oddsidemargin=0.1cm
\linespread{1.2}

\newcommand{\HRule}{\rule{\linewidth}{0.5mm}}
\DefineVerbatimEnvironment{code}{Verbatim}{frame=single, fontsize=\small}

\begin{document}

\begin{titlepage}

\begin{center}

\begin{tabular}{ l @{\hspace{1.2cm}}r }
\includegraphics[scale=0.3]{rhm.jpg} & \includegraphics[scale=0.9]{sibnigmi.jpg}\\[3cm]
\end{tabular}

% Title
\HRule \\[0.4cm]
{ \huge \bfseries KernelGen -- multi-target kernels generator for Fortran programs}\\[0.4cm]

\HRule \\[0.5cm]

\textsc{\Large User quick start guide}\\[1.5cm]


% Author and supervisor
\begin{minipage}{0.4\textwidth}
\begin{flushleft} \large
\end{flushleft}
\end{minipage}
\begin{minipage}{0.4\textwidth}
\begin{flushright}
\large Dmitry.~N.~Mikushin \\
{\small maemarcus@gmail.com} \\[0.5cm]
\end{flushright}
\end{minipage}

\vfill

% Bottom of the page
{\large \today}

\end{center}

\end{titlepage}

\section{Prerequisites}

KernelGen relies on other software and was tested on systems with 64-bit Fedora and Ubuntu. The lists of corresponding packages below are known to satisfy external dependencies:

\begin{itemize}
\item Fedora
\begin{itemize}
\item perl-XML-LibXSLT
\item perl-IPC-Run3
\item libffi-devel
\item gcc-gfortran
\end{itemize}
\item Debian / Ubuntu
\begin{itemize}
\item libxml-xslt-perl
\item libxml-libxslt-perl
\item libipc-run3-perl
\item libffi-dev
\item gfortran
\end{itemize}
\end{itemize}

\section{Supported hardware}

KernelGen supports kernels generation both for CUDA and OpenCL. In case of non-NVIDIA GPUs, CUDA will not be available, however OpenCL code generation should work well. Due to this segmentation, there are two different versions of KernelGen binaries: one is for CUDA+OpenCL and another -- for OpenCL only. If you have NVIDIA GPU, install package kernelgen-0.1-cuda.x86\_64.rpm, otherwise install package kernelgen-0.1-opencl.x86\_64.rpm.

\section{Installation}

Download kernelgen-0.1-cuda.x86\_64.rpm or kernelgen-0.1-opencl.x86\_64.rpm package from project files, depending on your target hardware. On Fedora install package using \emph{rpm} or \emph{yum}:

\begin{code}
[marcusmae@noisy x86_64]\$ sudo yum install kernelgen-0.1-cuda.x86_64.rpm
Loaded plugins: langpacks, presto, refresh-packagekit
Setting up Install Process
Examining kernelgen-0.1-cuda.x86_64.rpm: kernelgen-0.1-cuda.x86_64
Marking kernelgen-0.1-cuda.x86_64.rpm to be installed
Resolving Dependencies
--> Running transaction check
---> Package kernelgen.x86_64 0:0.1-cuda will be installed
--> Finished Dependency Resolution

Dependencies Resolved

===============================================================================
 Package              Arch      Version     Repository                   Size
===============================================================================
Installing:
 kernelgen            x86_64    0.1-cuda    /kernelgen-0.1-cuda.x86_64   54 M

Transaction Summary
===============================================================================
Install       1 Package(s)

Total size: 54 M
Installed size: 54 M
Is this ok [y/N]: y
Downloading Packages:
Running rpm_check_debug
Running Transaction Test
Transaction Test Succeeded
Running Transaction
  Installing : kernelgen-0.1-cuda.x86_64                                                                                                         1/1 

Installed:
  kernelgen.x86_64 0:0.1-cuda                                                                                                                        

Complete!
\end{code}

On Ubuntu systems you can use \emph{alien} first to convert .rpm package to .deb, and then install it in the same way with \emph{dpkg}.

KernelGen will be installed to \emph{/opt/kgen}, paths to compiler should be automatically added to shell profile, and dynamic libraries should be added to \emph{ldconfig} index.

\end{document}
